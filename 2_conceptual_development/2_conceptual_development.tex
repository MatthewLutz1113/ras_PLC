\section{Conceptual Development}

The heart of the project is the PLC (simulated by an Arduino). 

\subsection{Motors} 

There are three motors total in the project. One motor actuates the belt which linearly opens and closes the blinds, sliding the wand control along the rail. The second motor is attached to the wand control, where the wand would normally attach. The third motor operates in a similar mannar to the first, except that it is connected to the sliding screen door.

All three motors will need to have some level of torque and speed, to be determined, and will likely need a slightly larger power supply than that supplied on the Arduino. Thus, a power amplifier will be necessary.

\subsection{Calculating the Linear Actuation Motor Torque}

A force reader was used to calculated the forces required to open and close the blinds. The force meter was attached to the end of the blinds, and the maximum force before the blinds started moving was recorded for multiple positions along the length of the open and close duration. The maximum recorded force was then used to determine the size of the motor. The torque for the motor was then calculated using
\begin{align*}
    \tau = F \cdot d
\end{align*}

\subsection{Calculating the Rotational Actuation Motor Torque}

To simplify the process, the gear to be used on the blinds was created and attached to the wand control. Then a string was attached to the gear and then attached to the force reader, which was pulled and the force was read. The motor torque was once again computed from:
\begin{align*}
    \tau = F \cdot d
\end{align*}

\subsection{PLC}

The PLC recieves input from limit switches and other sensors on the blinds, and sends output to the Arduino. 

\subsection{Arduino}

The Arduino recieves input from the PLC as well as sensors, and sends output to power amplifiers.